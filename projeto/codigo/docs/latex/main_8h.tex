\hypertarget{main_8h}{}\section{include/main.h File Reference}
\label{main_8h}\index{include/main.\+h@{include/main.\+h}}


File containing example of doxygen usage for quick reference.  


{\ttfamily \#include $<$stdio.\+h$>$}\newline
{\ttfamily \#include \char`\"{}helpers.\+h\char`\"{}}\newline


\subsection{Detailed Description}
File containing example of doxygen usage for quick reference. 

\begin{DoxyAuthor}{Author}
A\+U\+T\+H\+OR N\+A\+ME 
\end{DoxyAuthor}
\begin{DoxyDate}{Date}
9 Sep 2012 Here typically goes a more extensive explanation of what the header defines. Doxygens tags are words preceeded by either a backslash \textbackslash{} or by an at symbol @. 
\end{DoxyDate}
\begin{DoxySeeAlso}{See also}
\href{http://www.stack.nl/~dimitri/doxygen/docblocks.html}{\tt http\+://www.\+stack.\+nl/$\sim$dimitri/doxygen/docblocks.\+html} 

\href{http://www.stack.nl/~dimitri/doxygen/commands.html}{\tt http\+://www.\+stack.\+nl/$\sim$dimitri/doxygen/commands.\+html} 
\end{DoxySeeAlso}
